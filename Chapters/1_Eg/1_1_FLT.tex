\section{Fermat's Last Theorem}

A classic example of this is Fermat's Last Theorem, which studies the Diophantine Equation
\begin{align}
    x^n + y^n = z^n
    \label{Ch1:eq:FLT}
\end{align}
where $x, y, z \in \Z$ and $n \in \N$. We motivate our study of algebraic number theory by studying the specific case of~\eqref{Ch1:eq:FLT} when $n = 2$.

\subsection{The Pythagorean Case: $n = 2$}

When $n = 2$, finding $x, y, z \in \Z$ such that~\eqref{Ch1:eq:FLT} holds is tantamount to finding Pythagorean triples $x, y, z$. We begin by making a few simplifications of the problem.

\begin{description}
    \item[\textbf{\underline{Reduction 1.}}] We can assume $x, y, z > 0$. Clearly, if $\parenth{x, y, z}$ is a solution, so is $\parenth{\pm x, \pm y, \pm z}$, for any $x, y, z \in \Z$. Furthermore, if one of the three variables is zero, the problem is easily solved, so we will disregard these cases.

    \item[\textbf{\underline{Reduction 2.}}] We can assume that $x, y, z$ are pairwise coprime. The reason for this is that if $p$ is any common prime divisor of $x, y, z$ then $\parenth{\frac{x}{p}, \frac{y}{p}, \frac{z}{p}} \in \Z^3$, and basic algebraic manipulation shows that this is still a solution. Indeed, if $p$ divides any two of them, then one can show, without much difficulty, that $p$ must also divide the third.

    \item[\textbf{\underline{Reduction 3.}}] For primitive solutions\footnote{ie, solutions where $x, y, z$ are pairwise coprime}, we can assume that not all $x, y, z$ are odd. The reason for this is that if both $x$ and $y$ are odd, then both $x$ and $y$ are congruent to either $1$ or $3$ modulo $4$. Either way, their square must be congruent to $1$ modulo $4$, making the sum of their squares congruent to $2$ modulo $4$. However, $2 \in \quotient{\Z}{4\Z}$ is not a square, meaning that if $x$ and $y$ are both odd, then there cannot be a $z$ satisfying the desired equation. Therefore, one of $x$ and $y$ must be even.
\end{description}

Given these reductions, the idea is to rewrite~\eqref{Ch1:eq:FLT} as
\begin{align*}
    y^2 = z^2 - x^2 = \parenth{z + x}\parenth{z - x}
\end{align*}
where we assume $x$ to be even (we need to assume that one of $x$ and $y$ is even, and it does not matter which of them we pick to be even).

First, we note that $\gcd\!\parenth{z + x, z - x} = 1$. Indeed, if $p$ divides both $z + x$ and $z - x$, then $p$ must divide $2z$, which is the sum of these two quantities. $p$ cannot divide $z$, as that would mean $p \vert x$, which contradicts the fact that $x$ and $z$ have no common factors. But $p$ cannot equal $2$, either, as that would again imply that $p$ divides $x$ and therefore that $p$ also divides $z$, resulting in the same contradiction. Therefore, $z + x$ and $z - x$ must be coprime.

We will now apply the fact that $\Z$ is a UFD. Since $x, z > 0$, it is immediate that $x + z > 0$. Furthermore, we know that $z^2 \geq x^2$, because $z^2$ is the sum of $x^2$ and a nonnegative quantity, namely, $y^2$. Then, since $x, z > 0$, we can conclude that $z > x$, meaning that $z - x > 0$. Finally, we know that the product of $z - x$ and $z + x$ is a square. From this, we can conclude that there exist $\alpha, \beta \in \Z$ such that $z - x = \alpha^2$ and $z + x = \beta^2$. We can therefore write
\begin{align*}
    x &= \frac{\parenth{z + x} - \parenth{z - x}}{2} = \frac{\beta^2 - \alpha^2}{2} \\
    z &= \frac{\parenth{z - x} + \parenth{z + z}}{2} = \frac{\beta^2 + \alpha^2}{2} \\
    y &= \alpha \beta \qquad \parenth{\text{because} y^2 = \alpha^2 \beta^2}
\end{align*}

\subsection{Generalising this Argument}

A crucial step in the above argument was the so-called \textbf{separating powers trick} for $n = 2$.

\begin{boxlemma}
    Let $a, b, c \in \Z \setminus \set{0}$, and let $n > 0$ be such that $a^n = b c$. If $b$ and $c$ are coprime, then there exist $b_1, c_1 \in \Z$ such that $b = \pm b_1 ^ n$ and $c = \pm c_1 ^ n$.
\end{boxlemma}

We can try and generalise this argument to when $n = p$ is an odd prime. Denote by $\zeta_p$ a primitive $p$th root of unity. We can factorise the expression $x^p + y^p$ as follows:
\begin{align*}
    z^p = x^p + y^p = \parenth{x + y}\parenth{x + \zeta_p y} \cdots \parenth{x + \zeta_p^{p - 1}y}
\end{align*}
The problem is, the factors of $z^p$ no longer lie in $\Z$: rather, they lie in the ring
\begin{align*}
    \Z\!\brac{\zeta_p} = \setst{\sum_{i=0}^{p-1}a_i \zeta_p^i}{a_i \in \Z}
\end{align*}
One can check that this is, in fact, a subring of $\C$. A very natural question we can ask ourselves is whether this is a UFD. Similarly, we can ask whether the separating powers trick applies in this setting as well. It turns out that it does in the case where $p$ is a \textbf{regular prime}, and we will see exactly what this means later on.

The goal of this module is to study rings "similar" to $\Z\!\brac{\zeta_p}$. We will eventually be more precise about what similarities we are interested in.