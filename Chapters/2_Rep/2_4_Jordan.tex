\section{Generalising the Additive Jordan Decomposition to Semi-Simple Lie Algebras}

We already know what Jordan Decompositions look like in the context of the general linear Lie algebra: it is simply a special decomposition of a linear map as the sum of a diagonal and a nilpotent linear map. In this section, we will explore a similar notion for elements of semi-simple algebras.

Throughout this section, we will assume that $L$ is semi-simple, and fix an arbitrary element $x \in L$. The main result that we will prove is the following.

\begin{boxtheorem}\label{Ch2:Thm:GenJordanDecomp}
    There is a unique decomposition
    \begin{align}
        x = d + n
        \label{Ch2:Eq:GenJordanDecomp}
    \end{align}
    where $d$ is diagonalisable and $n$ is nilpotent. In particular,
    \begin{enumerate}[label = \normalfont \arabic*., noitemsep]
        \item $\pad{d} \in \gl{L}$ is diagonalisable.
        \item $\pad{n} \in \gl{L}$ is nilpotent.
        \item $\brac{d, n} = 0$.
    \end{enumerate}
\end{boxtheorem}

We will prove existence and uniqueness separately. We will build on what we know about Jordan Decompositions from Linear Algebra.

Consider the adjoint $\pad{x} \in \gl{L}$. We know that $\pad{x}$ admits a Jordan decomposition $D + N$, with $D$ diagonalisable and $N$ nilpotent. These $D$ and $N$ satisfy the properties we would expect to have of $\pad{d}$ and $\pad{n}$, where $d$ and $n$ are as in~\eqref{Ch2:Eq:GenJordanDecomp} (if such elements exist). In particular, since $\ad : L \to \gl{L}$ is linear, we would have
\begin{align*}
    \pad{x} = \pad{d + n} = \pad{d} + \pad{n}
\end{align*}
We will use the same notation as above for the remainder of this section.

\subsection{Existence}

It suffices to show that $\pad{d} = D$ and $\pad{n} = N$. Before we can do this, we will need some machinery about $\pad{L}$, the image of $\ad : L \to \gl{L}$ in $\gl{L}$, and $\Der{L}$, the space of all derivations of $L$ (cf. \Cref{Ch1:Def:Derivation}).

\textbf{A number of the following are not true of when working in arbitrary Lie algebras. Many rely quite heavily on the semi-simplicity of $L$.}

We have already seen, in \Cref{Ch1:Lemma:adSubalgDer}, that $\pad{L} \leq \Der{L}$. It turns out that we can show something even stronger in the case of semi-simple Lie algebras.

\begin{lemma}
    $\pad{L} \nsg \Der{L}$.
\end{lemma}
\begin{proof}
    
    \sorry
\end{proof}

We have a similar (and less nontrivial) result about the relationship between $\pad{L}$ and $L$.

\begin{lemma}
    $\pad{L} \nsg L$.
\end{lemma}

\begin{lemma}
    $\pad{L}^{\perp} = \set{0}$, where we take the orthogonal complement with respect to the Killing Form $\kappa_{L}$.
\end{lemma}

The fact that $\pad{L}$ is an ideal of $L$ 

This allows us to 

\begin{boxproposition}
    $\Der{L} = \pad{L}$.
\end{boxproposition}
\begin{proof}
    
\end{proof}

\subsection{Uniqueness}

\subsection{Properties}

Throughout this subsection, fix $\delta \in \Der{L}$. Denote its additive Jordan Decomposition as $\delta = \nu + \mu$. Write
\begin{align*}
    L = \bigoplus_{\alpha} L_{\alpha}  % Over what does alpha range?
\end{align*}

\begin{definition}
    For some $\alpha \in \C$, define 
    \begin{align*}
        L_{\alpha} := \setst{v \in L}{\exists n \in \N \st \parenth{\delta - \alpha \cdot \id_L}^n v = 0}
    \end{align*}
\end{definition}

\begin{lemma}\label{Ch2:Lemma:BracGenEigenspJordan}
    For some $\alpha, \beta \in \C$, $\brac{L_{\alpha}, L_{\beta}} \subseteq L_{\alpha + \beta}$.
\end{lemma}
\begin{proof}
    Fix $x \in L_{\alpha}$ and $y \in L_{\beta}$. 
    \sorry
\end{proof}

\begin{lemma}
    $\nu, \mu \in \Der{L}$.
\end{lemma}
\begin{proof}
    Observe that $\nu\vert_{L_{\alpha}} = \alpha \cdot {\id}_{L_{\alpha}}$, ie, when restricted to $L_{\alpha}$, $\nu$ acts as $\alpha$ times the identity. Now, \sorry
\end{proof}


