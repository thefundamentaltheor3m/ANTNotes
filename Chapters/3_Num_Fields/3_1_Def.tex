\section{Definitions and Basic Properties}

\subsection{A Word on Field Extensions}

\begin{boxdefinition}[Finite Field Extension]
    We say a field extension $k \subseteq K$ is \textbf{finite} if the dimension of $K$ as a $k$-vector space is finite.
\end{boxdefinition}

\begin{boxdefinition}[Degree of a Field Extension]
    The dimension of an extension $K$ of a field $k$ as a $k$-vector space is called the \textbf{degree} of the field extension $K/k$, and is denoted $\brac{K : k}$.
\end{boxdefinition}

\subsection{Extensions of the Rational Numbers}

\begin{boxdefinition}[Number Field]
    A \textbf{number field} is a finite extension of the field of rational numbers $\Q$.
\end{boxdefinition}

The following is a well-known fact from Galois Theory that we will not prove in this module.

\begin{boxtheorem}[Primitive Element Theorem]\label{Ch2:Thm:Finite_Extensions_of_Q}
    Every finite extension of $\Q$ is of the form $\Q(\alpha)$ for some algebraic number $\alpha$.
    % Crazy!
\end{boxtheorem}

\subsection{Rings of Integers}

Throughout this subsection, let $k$ be a number field.

\begin{boxdefinition}[Ring of Integers]
    The \textbf{ring of integers} of a number field $k$ is the set of all algebraic integers contained in $k$.
\end{boxdefinition}

Note that by \Cref{Ch2:Thm:Finite_Extensions_of_Q}, this is a sensible definition: the algebraic integers live in $\C$, as does $\Q(\alpha)$ for any algebraic number $\alpha$, because $\alpha$ is a Complex root of a polynomial with rational coefficients.

\begin{boxproposition}
    $\Frac{\O_k} = k$.
\end{boxproposition}
\begin{proof}
    Note that the inclusion $\Frac{\O_k} \subseteq k$ is trivial, since $\O_k \subseteq k$ by definition. For the opposite inclusion, it is enough to show that $\forall \alpha \in k$, $\exists \lambda \in \Z$ such that $\lambda \cdot \alpha \in \O_k$. Indeed, this would imply that $\alpha = \frac{\lambda \alpha}{\lambda} \in \Frac{\O_k}$.

    Fix $\alpha \in k$. Since $k$ is a finite extension of $\Q$, there exists a polynomial $f \in \Q[X]$ such that $f(\alpha) = 0$. Let $f = \sum_{i=0}^n a_i X^i$. Then, $a_n \alpha^n + \cdots + a_0 = 0$. Define $\lambda$ to be the least common multiple of the denominators of $a_1, \ldots, a_n$. Then, \sorry
\end{proof}

\begin{boxproposition}
    $\O_k$ is integrally closed.
\end{boxproposition}
\begin{proof}
    We essentially need to show that for all monic polynomials $f \in \O_k[X]$ and $x \in \Frac{\O_k} = k$, if $f(\alpha) = 0$ then $\alpha \in \O_k$.

    But this follows quite naturally from \sorry
\end{proof}


