\section{Normed Rings}\label{Ch2:Sec:Norms}

\subsection{Norm Functions}

Let $R$ be a ring.

\begin{boxdefinition}[Norm]
    We say that $N : R \to \R$ is a \textbf{norm} if for all $\alpha, \beta \in R$,
    \begin{enumerate}[noitemsep]
        \item $N(\alpha) \geq 0$ and $N(\alpha) = 0 \iff \alpha = 0$.
        \item $N(\alpha \beta) = N(\alpha) N(\beta)$.
        \item $N(\alpha) = 1 \iff \alpha \in R^{\times}$.
    \end{enumerate}
\end{boxdefinition}

A normed ring is then exactly what we would expect it to be.

\begin{boxdefinition}[Normed Ring]
    We say that $R$ is a \textbf{normed ring} if there exists a norm function on $R$.
\end{boxdefinition}

There are many examples of normed rings that we have encountered before.

\begin{boxexample}
    $\Z$ is a normed ring under the Euclidean norm on $\R$ restricted to $\Z$.
\end{boxexample}

In the case of $\Z$, the Euclidean norm acts as a \textbf{size function} that we can use to show $\Z$ is a Eucldiean Domain. However, in general, admitting a norm does not make a ring a Euclidean Domain---indeed, there are normed rings that are not even UFDs. In the next few subsections, we will explore important normed rings.

\subsection{The Gaussian Integers}

\begin{boxdefinition}[The Gaussian Integers]
    The \textbf{Gaussian Integers} are the subring $\Zext{i}$ of $\C$, where $i = \sqrt{-1}$ is the imaginary constant.
\end{boxdefinition}

The Gaussian Integers are normed.

\begin{boxproposition}
    The function $N : \Zext{i} \to \R : a + bi \mapsto a^2 + b^2$ is a norm on $\Zext{i}$.
\end{boxproposition}

In fact, we can go a step further.

\begin{boxtheorem}\label{Ch1:Thm:GaussianIntegersEuclidean}
    Under the restriction of the above function $N$ to the set $\Zext{i} \setminus \set{0}$, the Gaussian Integers form a Euclidean Domain.
\end{boxtheorem}

The proof depends strongly on a geometric result that states that for any point in $\C$, there exists a point on $\Zext{i}$ that is at a distance of most $\frac{\sqrt{2}}{2}$ from it (cf. \cite[Proposition 2.4]{AmbrusPal}). For the proof of \Cref{Ch1:Thm:GaussianIntegersEuclidean}, see \cite[Theorem 2.3]{AmbrusPal}.

\subsection{The Rings $\Zext{i\sqrt{k}}$}

The Gaussian Integers are one example of a very broad family of rings.

\begin{boxdefinition}
    For $k \in \Z_{\geq 0}$, denote by $\sqrt{k}$ the \textit{positive} square root of $k$. Consider the set
    \begin{align}
        \Zext{i \sqrt{k}} = \setst{a + i\sqrt{k} \in \C}{a, b \in \Z}
    \end{align}
    where $i = \sqrt{-1}$ is the imaginary unit. It is easy to show that $\Zext{i\sqrt{k}}$ is a subring of $\C$. 
\end{boxdefinition}

