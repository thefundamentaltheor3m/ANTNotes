\section{Normed Rings}\label{Ch2:Sec:Norms}

\subsection{Norm Functions}

Let $R$ be a ring.

\begin{boxdefinition}[Norm]
    We say that $N : R \to \R$ is a \textbf{norm} if for all $\alpha, \beta \in R$,
    \begin{enumerate}[noitemsep]
        \item $N(\alpha) \geq 0$ and $N(\alpha) = 0 \iff \alpha = 0$.
        \item $N(\alpha \beta) = N(\alpha) N(\beta)$.
        \item $N(\alpha) = 1 \iff \alpha \in R^{\times}$.
    \end{enumerate}
\end{boxdefinition}

A normed ring is then exactly what we would expect it to be.

\begin{boxdefinition}[Normed Ring]
    We say that $R$ is a \textbf{normed ring} if there exists a norm function on $R$.
\end{boxdefinition}

There are many examples of normed rings that we have encountered before.

\begin{boxexample}
    $\Z$ is a normed ring under the Euclidean norm on $\R$ restricted to $\Z$.
\end{boxexample}

In the case of $\Z$, the Euclidean norm acts as a \textbf{size function} that we can use to show $\Z$ is a Eucldiean Domain. However, in general, admitting a norm does not make a ring a Euclidean Domain---indeed, there are normed rings that are not even UFDs. In the next few subsections, we will explore important normed rings.

\subsection{The Gaussian Integers}

\begin{boxdefinition}[The Gaussian Integers]
    The \textbf{Gaussian Integers} are the subring $\Zext{i}$ of $\C$, where $i = \sqrt{-1}$ is the imaginary constant.
\end{boxdefinition}



\subsection{The Rings $\Zext{i\sqrt{k}}$}


