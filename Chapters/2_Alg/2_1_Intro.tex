\section{Important Classes of Rings}

We begin by stating basic properties about rings.Let $R$ be a ring.

\begin{boxdefinition}[Group of Units]
    The \textbf{group of units} of $R$ is defined ot be the set
    \begin{align*}
        R^{\times} := \setst{a \in R}{\exists b \in R \st ab = 1}
    \end{align*}
    which forms a group under the ring's multiplication operation. Elements of $R^{\times}$ are called \textbf{units}.
\end{boxdefinition}

The first property of the integers we can generalise is that of divisors.

\begin{boxdefinition}
    For every $a, b \in R$, we say that \textbf{$a$ divides $b$} if $\exists c \in R$ such that $ac = b$. We denote this property as $a \vert b$.
\end{boxdefinition}

\begin{boxdefinition}[Irreducibility]
    We say that $a \in R \setminus \set{0}$ is \textbf{irreducible} if $a$ is not a unit and for every $b, c \in R$ such that $a = bc$, we have that either $b$ or $c$ is a unit.
\end{boxdefinition}

\subsection{Integral Domains}

For the remainder of this subsection, assume that $R$ is an integral domain.

\begin{boxdefinition}[Prime Elements]
    We say an element $a \in R \setminus \set{0}$ is prime if $a$ is not a unit and if for all $b, c \in R$, if $a \vert bc$ then either $a \vert b$ or $a \vert c$.
\end{boxdefinition}

We have the following relationship between irreducible and prime elements.

\begin{boxlemma}\label{Ch2:Lemma:PrimeIrred}
    Every prime element of $R$ is irreducible.
\end{boxlemma}

\begin{boxexample}\label{Ch2:Eg:Z[i sqrt(k)]}
    For $k \in \Z_{\geq 0}$, denote by $\sqrt{k}$ the \textit{positive} square root of $k$. Consider the set
    \begin{align}
        \Zext{i \sqrt{k}} = \setst{a + i\sqrt{k} \in \C}{a, b \in \Z}
    \end{align}
    where $i = \sqrt{-1}$ is the imaginary unit. It is easy to show that $\Zext{i\sqrt{k}}$ is a subring of $\C$.
\end{boxexample}

\subsection{Unique Factorisation Domains}

A UFD is exactly what it sounds like.

\begin{boxdefinition}[Unique Factorisation Domain]
    We say that an integral domain $R$ is a \textbf{unique factorisation domain}, or a \textbf{UFD}, if every element of $R$ is uniquely expressible as a product of prime elements, up to reordering of primes and multiplication by a unit.
\end{boxdefinition}

In a UFD, the converse of \Cref{Ch2:Lemma:PrimeIrred} is true. We sketch the proof below.

\begin{boxlemma}
    If $R$ is a UFD, then every irreducible element of $R$ is prime.
\end{boxlemma}
\begin{proof}
    Assume that $a \in R \setminus \set{0}$ is irreducible. Let $b, c \in R$ be such that $a \vert bc$. Then, there exists some $d \in R$ such that $ad = bc$. We can then use the fact that $ad$ and $bc$ must admit the same unique factorisation into irreducibles to see that $a$ must appear in the unique factorisation of $bc$, making it a divisor of either $b$ or $c$.
\end{proof}

We give a non-trivial example of a ring that is \textbf{not} a unique factorisation domain in \Cref{Ch2:Sec:Norms}.

\subsection{Principal Ideal Domains}

PIDs, too, are exactly what they sound like.

\begin{boxdefinition}[Principal Ideal Domain]
    An integral domain $R$ is called a \textbf{Principal Ideal Domain}, or \textbf{PID}, if every ideal of $R$ is generated by a single element.
\end{boxdefinition}

We have an important relationship between PIDs and UFDs.

\begin{boxtheorem}
    Every PID is a UFD.
\end{boxtheorem}

For the proof of this result, see \cite[Theorem 1.2]{AmbrusPal}.

\subsection{Euclidean Domains}

Euclidean Domains are essentially Integral Domains in which we can perform long division using the Euclidean Algortihm.

\begin{boxdefinition}[Euclidean Domain]
    \sorry % What's the point, we know this stuff
\end{boxdefinition}

There are many examples of Euclidean Domains.

\begin{boxexample}[Polynomial Rings over Fields]
    Let $k$ be a field. Then, the polynomial ring $k[X]$ is a Euclidean Domain with size function $2^{\deg}$ (or $n^{\deg}$ for any $n \in \N_{>1}$).
\end{boxexample}

We have an important relationship between Euclidean Domains and PIDs (and, by extension, UFDs).

\begin{boxproposition}
    Every Euclidean Domain is a PID.
\end{boxproposition}

We again skip the proof and refer the reader to \cite{}

\subsection{Integrally Closed Rings}

In this subsection, we will assume nothing about $R$ beyond that it is an integral domain. We denote its field of fractions by $K = \Frac{R}$.

\begin{boxdefinition}[Integrally Closed]
    $R$ is \textbf{integrally closed} if for every monic polynomial $f \in R[X]$ and every $\frac{a}{b} \in K$, we have the implication
    \begin{align*}
        \fof{\frac{a}{b}} = 0 \implies \frac{a}{b} \in K
    \end{align*}
\end{boxdefinition}

It turns out there is a simple criterion for being integrally closed.

\begin{boxproposition}
    If $R$ is a UFD, then $R$ is integrally closed.
\end{boxproposition}
\begin{proof}
    Let $f \in R[X]$ be a monic polynomial. Write
    \begin{align*}
        \fof{X} = X^n + a_{n-1} X^{n-1} + \cdots + a_0
    \end{align*}
    Fix some $\frac{\alpha}{\beta} \in K$. Assume that $\fof{\frac{\alpha}{\beta}} = 0$. We can assume, without loss of generality, that $\alpha$ and $\beta$ have no common factors that are not units: if they do, then we can write out both of their \textit{unique} factorisations into irreducibles and perform the necessary cancellations in $K$ to reduce $\frac{\alpha}{\beta}$ to a fraction whose numerator and denominator have no common factors.

    We have that
    \begin{align*}
        \frac{\alpha^n}{\beta^n} &= -\parenth{a_{n-1}\frac{\alpha^{n-1}}{\beta^{n-1}} + \cdots + a_0} \\
        \implies \alpha^n &= -\beta\parenth{a_{n-1}\alpha^{n-1} + \cdots + a_0 \beta^{n-1}}
    \end{align*}
\end{proof}