\section{Propositional Formulae}

\subsection{Propositions and Connectives}

We begin by defining the notion of a proposition.

\begin{boxdefinition}[Proposition]
    A proposition is a statement that is either true or false.
\end{boxdefinition}

\begin{boxconvention}
    We will denote the state of being \textbf{true} by $\T$ and the state of being \textbf{false} by $\F$.
\end{boxconvention}

Propositions can be connected to each other using tools known as \textbf{connectives}. These can be thought of as \textbf{truth table rules}.

\begin{boxconvention}
    Before we define the actual connectives we shall use, we list them down, along with notation.
    \begin{enumerate}[noitemsep]
        \item Conjunction
        \item Disjunction
        \item Negation
        \item Implication
        \item The Biconditional
    \end{enumerate}
    In particular, we will only use the $\implies$ and $\iff$ symbols when reasoning \textbf{informally}. For \textbf{formal} use, we will stick to the $\to$ and $\lr$ symbols.
\end{boxconvention}

We define them comprehensively as follows.
\begin{boxdefinition}[Connectives]\label{Ch1:Def:Connective}
    \sorry % Fill in truth table
\end{boxdefinition}

We are now ready to define the main object of study in this section: propositional formulae.

\begin{boxdefinition}[Propositional Formula]\label{Ch1:Def:PropFormula}
    A propositional formula is obtained from propositional variables $P_1, P_2, \ldots$ and connectives via the following rules:
    \begin{enumerate}[label=\normalfont (\roman*), noitemsep]
        \item Any propositional variable is a propositional formula.
        \item If $\phi$ and $\psi$ are formulae, then so are $\neg \phi$, $\neg \psi$, $\phi \land \psi$, $\phi \lor \psi$, $\phi \to \psi$, $\psi \to \phi$, and $\phi \lr \psi$.
        \item Any formula arises in this manner after a finite number of steps.
    \end{enumerate}
\end{boxdefinition}

What this means informally is that a propositional formula is a string of symbols consisting of
\begin{enumerate}
    \item variables that take on true/false values,
    \item connectors that express the relationship between these variables, and
    \item parentheses/brackets that expresses the order in which the expression is to be evaluated when the variables are given specific values.
\end{enumerate}

What the three rules in \Cref{Ch1:Def:PropFormula} tells us is that every propositional formula is either a propositional variable or is built from `shorter' formulae, where by `shorter' we mean consisting of fewer symbols. Just as we use trees to evaluate expressions on the computer when performing arithmetic, we can use them to express and evaluate propositional formulae as well.

Furthermore, any assignment of truth values to the propositional variables in a formula $\phi$ determines the truth value for $\phi$ in a \textbf{unique} manner, using the comprehensive definitions of the connectives given in \Cref{Ch1:Def:Connective}.

\begin{boxexample}
    Consider the formula $\phi : \parenth{\parenth{P \to \parenth{\neg Q}} \to P}$. \sorry % Construct a truth table
\end{boxexample}

In the next section, we will be more precise about what evaluating a formula means.

\subsection{Truth Functions}

Throughout this subsection, let $n$ denote an arbitrary natural number.

\begin{boxdefinition}[Truth Function]
    A truth function of $n$ variables is a function
    \begin{align*}
        f : \set{\T, \F}^n \to \set{\T, \F}
    \end{align*}
\end{boxdefinition}

The point of a truth function is that we can relate them to propositional formulae. Let $\phi$ be a propositional formula whose variables are $p_1, \ldots, p_n$. We can associate to $\phi$ a truth function whose truth value at any $\parenth{x_1, \ldots, x_n} \in \set{\T, \F}^n$ corresponds to the truth value of $\phi$ that arises from setting $p_i$ to $x_i$ for all $1 \leq i \leq n$. 
